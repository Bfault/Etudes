\documentclass[a4paper, 12pt]{article}
\usepackage{amsmath, amssymb, amsthm}
\usepackage{geometry}
\usepackage{tcolorbox}
\geometry{hmargin=2.5cm, vmargin=2.5cm}

\renewcommand*{\today}{12 septembre 2024}

\title{Analyse | CM: 1}
\author{Par Lorenzo}
\date{\today}

\newtheorem{theorem}{Théorème}[section]
\newtheorem{definition}{Définition}[section]
\newtheorem{example}{Example}[section]
\newtheorem{remark}{Remarques}[section]
\newtheorem{lemme}{Lemme}[section]

\newtheorem{_proposition}{Proposition}[section]
\newenvironment{proposition}[1][]{
    \begin{_proposition}[#1]~\par
    \vspace*{0.5em}
}{%
    \end{_proposition}%
}

\newenvironment{rdem}[1][]{
    \begin{tcolorbox}[colframe=black, colback=white!10, sharp corners]
        #1
}{%
    \end{tcolorbox}
     
}

\newtheorem{_demonstration}{Démonstration}[section]
\newenvironment{demonstration}[1][]{
    \begin{_demonstration}[#1]~\par
    \vspace*{0.5em}
}{%
    \end{_demonstration}%
    \qed%
}

\newtheorem*{_demonstration*}{Démonstration}
\newenvironment{demonstration*}[1][]{
    \begin{_demonstration*}[#1]~\par
    \vspace*{0.5em}
}{%
    \end{_demonstration*}%
    \qed%
}

\newenvironment{proprietes}{
    \noindent\textbf{Propriétés}
    \begin{enumerate}
}{
    \end{enumerate}
}

\newenvironment{ldefinition}{
    \begin{definition}~\par
    \vspace*{0.5em}
    \begin{enumerate}
}{
        \end{enumerate}
        \end{definition}
}

\newenvironment{lexample}{
    \begin{example}~\par
    \vspace*{0.5em}
    \begin{enumerate}
}{
        \end{enumerate}
        \end{example}
}

\newtheorem{_methode}{Méthode}[section]
\newenvironment{methode}{
    \begin{_methode}~\par
    \vspace*{0.5em}
}{
        \end{_methode}
}

\def\N{\mathbb{N}}
\def\Z{\mathbb{Z}}
\def\Q{\mathbb{Q}}
\def\R{\mathbb{R}}
\def\C{\mathbb{C}}
\def\K{\mathbb{K}}

\def\un{(u_n)_{n \in \N}}
\def\xn#1{(#1_n)_{n \in \N}}

\def\o{\overline}
\def\eps{\varepsilon}

\newcommand{\lt}{\ensuremath <}
\newcommand{\gt}{\ensuremath >}

\begin{document}

\maketitle

\section{L'ensemble des nombres rationnels $\Q$}

\subsection{Ecriture décimale}

\begin{definition}
    On definit l'ensemble des nombres rationnels $\Q$ par \break
    $\Q = \{\dfrac{p}{q} \mid p \in \Z, q \in \N^*\}$,
    optionellement $pgcd(p, q) = 1$ peut être rajouté dans la définition.
    Cela ajoute le fait que p et q sont premier entre eux et donc $\dfrac{p}{q}$ un fraction irréductible.
    (rappel: $\N^*=\N\backslash\{0\}$, i.e. $\N$ privé de 0).
\end{definition}

\begin{remark}
    Les nombres décimaux sont des nombres de la forme \break $\dfrac{p}{10^n}$ avec $p \in \Z, n \in \N$
    (e.g. $1.234 = \dfrac{1234}{10^3}$).
\end{remark}

\begin{proposition}
    Un nombre est rationnel si et seulement si il admet une écriture décimale finie ou périodique.
\end{proposition}

\begin{example}
    Prenons $x = 12.34202320232023...$

    \vspace{1em}

    Etape 1: faire apparaitre la partie périodique à la virgule.
    Ici on multiplie par 100
    \begin{flalign}
        100x = 1\,234.202320232023...&&
    \end{flalign}
    Etape 2: on décale d'une période vers la gauche.
    Ici la période est de longeur 4 donc on multiplie par 10 000.
    \begin{flalign}
        10\,000 \times 100x = 12\,342\,023.20232023...&&
    \end{flalign}
    Etape 3: on soustrait (2) par (1) pour que la partie décimale s'annule.
    \begin{flalign}
        &10\,000 \times 100x - 100x = 12\,342\,023 - 1\,234&& \\
        \iff &999\,900x = 12\,340\,789&& \\
        \iff &x = \dfrac{12\,340\,789}{999\,900}&& \\
    \end{flalign}
\end{example}

\end{document}
